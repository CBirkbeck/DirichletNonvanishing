% In this file you should put the actual content of the blueprint.
% It will be used both by the web and the print version.
% It should *not* include the \begin{document}
%
% If you want to split the blueprint content into several files then
% the current file can be a simple sequence of \input. Otherwise It
% can start with a \section or \chapter for instance.


\section{What this project is about}

We're trying to prove that for a non-trivial Dirichlet character $\chi$, we have $L(\chi, 1 + it) \ne 0$ for all real $t$.

\section{High-level outline}:

\begin{itemize}

\item If $t \ne 0$ or if $\chi^2 \ne 1$ then this follows from the product for $\zeta(s)^3 L(\chi, s)^{-4} L(\chi^2, s)$ which is in \texttt{EulerProducts}. So we may assume that $t = 0$ and $\chi$ is quadratic.

\item Assume for contradiction $L(\chi, 1) = 0$. Then the function
\[ F(s) = L(\chi, s) \zeta(s) \]
is entire.

\item For $\Re(s) > 1$, $F(s)$ is given by a convergent Euler product with local factors XXX. Hence its Dirichlet series coefficients are positive real numbers.

\item Hence the iterated derivatives of $F(s)$ on $(0, \infty)$ alternate in sign.

\item By an analytic result from \texttt{EulerProducts}, this implies that $F(s)$ is real and non-vanishing for all real $s$.


\item However, if $\chi$ is even then $F(0) = 0$, and if $\chi$ is odd then $F(-1) = 0$. This gives the desired contradiction.
\end{itemize}

\section{A more detailed plan}

\begin{lemma} \label{product-lower-bound}
  \lean{norm_dirichlet_product_ge_one}
  \leanok
  Let $\chi$ be a Dirichlet character modulo~$N$. Then for all $\varepsilon > 0$, we have
  \begin{equation} \label {eqn:product}
    \||L(1, 1 + \eps)^3 L(\chi, 1 + \eps + it)^4 L(\chi^2, 1 + \eps + 2it)\|| \ge 1 \,.
  \end{equation}
\end{lemma}

\begin{proof}
  \leanok
  This follows from a trigonometric inequality.
\end{proof}

\begin{lemma} \label{non-quadratic}
  Let $t \in \R$ and let $\chi$ be a Dirichlet character. If $t \ne 0$ or $\chi^2 \ne 1$, then
  \[ L(\chi, 1 + it) \ne 0 \,. \]
\end{lemma}

\begin{proof}
  \uses{product-lower-bound}
  Assume that $L(\chi, 1 + it) = 0$. Then the (at lesat) quadruple zero of~$L(\chi, s)^4$ at $1 + it$
  will more than compensate for the triple pole of~$L(1, s)^3$ at~$1$, so the product of the first
  two factors in~\eqref{eqn:product} will tend to zero as $\eps \searrow 0$.
  If $t \ne 0$ or $\chi^2 \ne 1$, then the last factor will have a finite limit, and so the
  full product will converge to zero, contradicting Lemma~\ref{product-lower-bound}.
\end{proof}


 So it suffices to prove that if $\chi$ is a quadratic character we have $L(\chi, 1) \ne 0$. Suppose, for contradiction, that $\chi$ is such a character with $L(\chi, 1) = 0$. (XXX Does it help if we assume $\chi$ primitive mod $N$ as well?)

 \begin{definition}
  Define $F : \mathbb{C} \to \mathbb{C}$ by
  \[ F(s) = \begin{cases}
   \zeta(s) L(\chi, s) & \text{if $s \ne 1$} \\
   L'(\chi, 1) & \text{if $s = 1$}
   \end{cases}
  \]
 \end{definition}

 \begin{lemma}
  $F$ is an entire function.
 \end{lemma}

 \begin{proof}
  This is easy for $s \ne 1$ since we know analyticity of both factors. To prove analyticity at $s = 1$, it suffices to show continuity (Riemann criterion) and that should follow easily since we know that $\lim_{s \to 1} (s - 1) \zeta(s)$ exists.
 \end{proof}

 \begin{lemma}
  We have $F(-2) = 0$.
 \end{lemma}

 \begin{proof}
  Follows from the analyticity of $L(\chi, s)$ and the trivial zeroes of Riemann zeta.
 \end{proof}