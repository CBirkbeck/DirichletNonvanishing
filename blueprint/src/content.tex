% In this file you should put the actual content of the blueprint.
% It will be used both by the web and the print version.
% It should *not* include the \begin{document}
%
% If you want to split the blueprint content into several files then
% the current file can be a simple sequence of \input. Otherwise It
% can start with a \section or \chapter for instance.

\setcounter{section}{0}

\section{What this project is about}

We're trying to prove the following theorem:

% \begin{theorem*}
%  \lean{ourMainTheorem}
\emph{Let $N \ge 1$ and $\chi$ a complex-valued Dirichlet character mod $N$. Then $L(\chi, 1 + it) \ne 0$ for all real $t$.}
% \end{theorem*}

See Theorem~\ref{dirichlet_char_nonvanishing} below.

\section{High-level outline}

\begin{itemize}

\item If $t \ne 0$ or if $\chi^2 \ne 1$ then this follows from the product for $\zeta(s)^3 L(\chi, s)^{-4} L(\chi^2, s)$ which is in \texttt{EulerProducts}. So we may assume that $t = 0$ and $\chi$ is quadratic.

\item Assume for contradiction $L(\chi, 1) = 0$. Then the function
\[ F(s) = L(\chi, s) \zeta(s) \]
is entire.

\item For $\Re(s) > 1$, $F(s)$ is given by a convergent Euler product with local factors XXX. Hence its Dirichlet series coefficients are positive real numbers.

\item Hence the iterated derivatives of $F(s)$ on $(0, \infty)$ alternate in sign.

\item By an analytic result from \texttt{EulerProducts}, this implies that $F(s)$ is real and non-vanishing for all real $s$.

\item However, $F(-2) = 0$. This gives the desired contradiction.

\end{itemize}

\section{A more detailed plan}

\begin{lemma} \label{product_lower_bound}
  \lean{norm_dirichlet_product_ge_one}
  \leanok
  Let $\chi$ be a Dirichlet character modulo~$N$. Then for all $\varepsilon > 0$, we have
  \begin{equation} \label {eqn:product}
    |L(1, 1 + \eps)^3 L(\chi, 1 + \eps + it)^4 L(\chi^2, 1 + \eps + 2it)| \ge 1 \,.
  \end{equation}
\end{lemma}

\begin{proof}
  \leanok
  This follows from a trigonometric inequality.
\end{proof}

\begin{lemma} \label{non_quadratic}
  \lean{mainTheorem_general}
  \leanok
  Let $t \in \R$ and let $\chi$ be a Dirichlet character. If $t \ne 0$ or $\chi^2 \ne 1$, then
  \[ L(\chi, 1 + it) \ne 0 \,. \]
\end{lemma}

\begin{proof}
  \uses{product_lower_bound}
  \leanok
  Assume that $L(\chi, 1 + it) = 0$. Then the (at lesat) quadruple zero of~$L(\chi, s)^4$ at $1 + it$
  will more than compensate for the triple pole of~$L(1, s)^3$ at~$1$, so the product of the first
  two factors in~\eqref{eqn:product} will tend to zero as $\eps \searrow 0$.
  If $t \ne 0$ or $\chi^2 \ne 1$, then the last factor will have a finite limit, and so the
  full product will converge to zero, contradicting Lemma~\ref{product_lower_bound}.
\end{proof}


So it suffices to prove that if $\chi$ is a quadratic character we have $L(\chi, 1) \ne 0$.

\begin{definition} \label{def:bad_char}
  \lean{BadChar}
  \leanok
  A \emph{bad character} is an $\R$-valued (hence quadratic) Dirichlet character such that $L(\chi, 1) = 0$.
\end{definition}

\begin{definition} \label{def:bad_char_F}
  \lean{BadChar.F}
  \leanok
  \uses{def:bad_char}
  Define $F \colon \mathbb{C} \to \mathbb{C}$ by
  \[ F(s) = \begin{cases}
    \zeta(s) L(\chi, s) & \text{if $s \ne 1$} \\
    L'(\chi, 1) & \text{if $s = 1$}
    \end{cases}
  \]
\end{definition}

\begin{lemma} \label{F_entire}
  \lean{BadChar.F_differentiable}
  \leanok
  \uses{def:bad_char_F}
  If $\chi$ is a bad character, then $F$ is an entire function.
\end{lemma}

\begin{proof}
  \leanok
  This is easy for $s \ne 1$ since we know analyticity of both factors. To prove analyticity at $s = 1$, it suffices to show continuity (Riemann criterion) and that should follow easily since we know that $\lim_{s \to 1} (s - 1) \zeta(s)$ exists.
\end{proof}

\begin{lemma}
 \label{zero_of_F}
 \lean{BadChar.F_neg_two}
 \leanok
 We have $F(-2) = 0$.
\end{lemma}

\begin{proof}
 \leanok
 \uses{def:bad_char_F}
 Follows from the trivial zeroes of Riemann zeta.
\end{proof}

\begin{lemma} \label{F_Euler_product}
  \uses{def:bad_char_F}
  \lean{BadChar.F_eq_LSeries}
  \leanok
  For $\Re(s) > 1$, $F(s)$ is equal to the $L$-series of a real-valued
  arithmetic function $e$ defined as the convolution of constant 1 and $\chi$.
\end{lemma}

\begin{proof}
  \leanok
  We have Euler products for both $L(\chi, s)$ and $\zeta(s)$.
\end{proof}

\begin{lemma}
  \label{nonneg_coeffs}
  \lean{BadChar.e_nonneg}
  \leanok
  The weakly multiplicative function $e(n)$ whose Euler product is $\mathcal{E}$ takes non-negative real values.
\end{lemma}

\begin{proof}
  \uses{F_Euler_product}
  \leanok
  It suffices to show this for prime powers. We have $e(p^k) = (k + 1)$ if $\chi(p) = 1$, $e(p^k) = 1$ if $\chi(p) = 0$, and if $\chi(p) = -1$ then $e(p^k) = 1$ if $k$ even, $0$ if $k$ odd.
\end{proof}

\begin{lemma} \label{positivity_from_derivs}
  \lean{Complex.at_zero_le_of_iteratedDeriv_alternating}
  \leanok
  An entire function $f$ whose iterated derivatives at zero are all real with alternating signs
  (except possibly the value itself) has values of the form $f(0) + \text{nonneg. real}$
  along the nonpositive real axis.
\end{lemma}

\begin{proof}
  \leanok
  This follows by considering the power series expansion at zero.
\end{proof}

\begin{lemma} \label{derivs_from_coeffs}
  If $a \colon \N \to \C$ is an arithmetic function with $a(1) > 0$ and $a(n) \ge 0$ for all $n \ge 2$
  and the associated $L$-series $f(s) = \sum_{n \ge 1} a(n) n^{-s}$ converges at $x \in \R$, then
  the iterated derivative $f^{(m)}(x)$ is nonnegative for $m$ even and nonpositive for $m$ odd.
  \leanok
\end{lemma}

\begin{proof}
  The $m$th derivative of~$f$ at~$x$ is given by
  \[ f^{(m)}(x) = \sum_{n=1}^\infty (-\log n)^m a(n) n^{-x} = (-1)^m \sum_{n=1}^\infty (\log n)^m a(n) n^{-x} \,, \]
  and the last sum has only nonnegative summands.
\end{proof}

\begin{lemma} \label{quadratic_char_nonvanishing}
  \lean{mainTheorem_quadratic}
  \leanok
  If $\chi$ is a nontrivial quadratic Dirichlet character, then $L(\chi, 1) \ne 0$.
\end{lemma}

\begin{proof}
  \uses{F_entire, zero_of_F, F_Euler_product, nonneg_coeffs, positivity_from_derivs, derivs_from_coeffs}
  Assume that $L(\chi, 1) = 0$, so $\chi$ is a bad character. By Lemma~\ref{F_entire}, we then know
  that $F$ is an entire function. From Lemmas \ref{F_Euler_product} and~\ref{nonneg_coeffs} we see
  that $F$ agrees on $\Re s > 1$ with the $L$-series of an arithmetic function with nonnegative
  real values (and positive value at~$1$). Lemma~\ref{derivs_from_coeffs} now shows that
  $(-1)^m F^{(m)}(2) \ge 0$ for all~$m \ge 1$. Then Lemma~\ref{positivity_from_derivs}
  (applied to $f(s) = F(2+s)$) implies that $F(x) > 0$ for all $x \le 2$. This now
  contradicts Lemma~\ref{zero_of_F}, which says that $F(-2) = 0$. So the initial assumption
  must be false, showing that $L(\chi, 1) \ne 0$.
\end{proof}

\begin{theorem} \label{dirichlet_char_nonvanishing}
  \lean{ourMainTheorem}
  \leanok
  If $\chi$ is a Dirichlet character and $t$ is a real number such that $t \ne 0$
  or $\chi$ is nontrivial, then $L(\chi, 1 + it) \ne 0$.
\end{theorem}

\begin{proof}
  \leanok
  \uses{non_quadratic, quadratic_char_nonvanishing}
  If $\chi$ is not a quadratic character or $t \ne 0$, then the claim is Lemma~\ref{non_quadratic}.
  If $\chi$ is a nontrivial quadratic characters and $t = 0$, then the claim is Lemma~\ref{quadratic_char_nonvanishing}.
\end{proof}
